% Options for packages loaded elsewhere
\PassOptionsToPackage{unicode}{hyperref}
\PassOptionsToPackage{hyphens}{url}
%
\documentclass[
]{article}
\usepackage{amsmath,amssymb}
\usepackage{lmodern}
\usepackage{iftex}
\ifPDFTeX
  \usepackage[T1]{fontenc}
  \usepackage[utf8]{inputenc}
  \usepackage{textcomp} % provide euro and other symbols
\else % if luatex or xetex
  \usepackage{unicode-math}
  \defaultfontfeatures{Scale=MatchLowercase}
  \defaultfontfeatures[\rmfamily]{Ligatures=TeX,Scale=1}
\fi
% Use upquote if available, for straight quotes in verbatim environments
\IfFileExists{upquote.sty}{\usepackage{upquote}}{}
\IfFileExists{microtype.sty}{% use microtype if available
  \usepackage[]{microtype}
  \UseMicrotypeSet[protrusion]{basicmath} % disable protrusion for tt fonts
}{}
\makeatletter
\@ifundefined{KOMAClassName}{% if non-KOMA class
  \IfFileExists{parskip.sty}{%
    \usepackage{parskip}
  }{% else
    \setlength{\parindent}{0pt}
    \setlength{\parskip}{6pt plus 2pt minus 1pt}}
}{% if KOMA class
  \KOMAoptions{parskip=half}}
\makeatother
\usepackage{xcolor}
\IfFileExists{xurl.sty}{\usepackage{xurl}}{} % add URL line breaks if available
\IfFileExists{bookmark.sty}{\usepackage{bookmark}}{\usepackage{hyperref}}
\hypersetup{
  pdftitle={Auto-Stats},
  pdfauthor={Arne John},
  hidelinks,
  pdfcreator={LaTeX via pandoc}}
\urlstyle{same} % disable monospaced font for URLs
\usepackage[margin=1in]{geometry}
\usepackage{color}
\usepackage{fancyvrb}
\newcommand{\VerbBar}{|}
\newcommand{\VERB}{\Verb[commandchars=\\\{\}]}
\DefineVerbatimEnvironment{Highlighting}{Verbatim}{commandchars=\\\{\}}
% Add ',fontsize=\small' for more characters per line
\usepackage{framed}
\definecolor{shadecolor}{RGB}{248,248,248}
\newenvironment{Shaded}{\begin{snugshade}}{\end{snugshade}}
\newcommand{\AlertTok}[1]{\textcolor[rgb]{0.94,0.16,0.16}{#1}}
\newcommand{\AnnotationTok}[1]{\textcolor[rgb]{0.56,0.35,0.01}{\textbf{\textit{#1}}}}
\newcommand{\AttributeTok}[1]{\textcolor[rgb]{0.77,0.63,0.00}{#1}}
\newcommand{\BaseNTok}[1]{\textcolor[rgb]{0.00,0.00,0.81}{#1}}
\newcommand{\BuiltInTok}[1]{#1}
\newcommand{\CharTok}[1]{\textcolor[rgb]{0.31,0.60,0.02}{#1}}
\newcommand{\CommentTok}[1]{\textcolor[rgb]{0.56,0.35,0.01}{\textit{#1}}}
\newcommand{\CommentVarTok}[1]{\textcolor[rgb]{0.56,0.35,0.01}{\textbf{\textit{#1}}}}
\newcommand{\ConstantTok}[1]{\textcolor[rgb]{0.00,0.00,0.00}{#1}}
\newcommand{\ControlFlowTok}[1]{\textcolor[rgb]{0.13,0.29,0.53}{\textbf{#1}}}
\newcommand{\DataTypeTok}[1]{\textcolor[rgb]{0.13,0.29,0.53}{#1}}
\newcommand{\DecValTok}[1]{\textcolor[rgb]{0.00,0.00,0.81}{#1}}
\newcommand{\DocumentationTok}[1]{\textcolor[rgb]{0.56,0.35,0.01}{\textbf{\textit{#1}}}}
\newcommand{\ErrorTok}[1]{\textcolor[rgb]{0.64,0.00,0.00}{\textbf{#1}}}
\newcommand{\ExtensionTok}[1]{#1}
\newcommand{\FloatTok}[1]{\textcolor[rgb]{0.00,0.00,0.81}{#1}}
\newcommand{\FunctionTok}[1]{\textcolor[rgb]{0.00,0.00,0.00}{#1}}
\newcommand{\ImportTok}[1]{#1}
\newcommand{\InformationTok}[1]{\textcolor[rgb]{0.56,0.35,0.01}{\textbf{\textit{#1}}}}
\newcommand{\KeywordTok}[1]{\textcolor[rgb]{0.13,0.29,0.53}{\textbf{#1}}}
\newcommand{\NormalTok}[1]{#1}
\newcommand{\OperatorTok}[1]{\textcolor[rgb]{0.81,0.36,0.00}{\textbf{#1}}}
\newcommand{\OtherTok}[1]{\textcolor[rgb]{0.56,0.35,0.01}{#1}}
\newcommand{\PreprocessorTok}[1]{\textcolor[rgb]{0.56,0.35,0.01}{\textit{#1}}}
\newcommand{\RegionMarkerTok}[1]{#1}
\newcommand{\SpecialCharTok}[1]{\textcolor[rgb]{0.00,0.00,0.00}{#1}}
\newcommand{\SpecialStringTok}[1]{\textcolor[rgb]{0.31,0.60,0.02}{#1}}
\newcommand{\StringTok}[1]{\textcolor[rgb]{0.31,0.60,0.02}{#1}}
\newcommand{\VariableTok}[1]{\textcolor[rgb]{0.00,0.00,0.00}{#1}}
\newcommand{\VerbatimStringTok}[1]{\textcolor[rgb]{0.31,0.60,0.02}{#1}}
\newcommand{\WarningTok}[1]{\textcolor[rgb]{0.56,0.35,0.01}{\textbf{\textit{#1}}}}
\usepackage{graphicx}
\makeatletter
\def\maxwidth{\ifdim\Gin@nat@width>\linewidth\linewidth\else\Gin@nat@width\fi}
\def\maxheight{\ifdim\Gin@nat@height>\textheight\textheight\else\Gin@nat@height\fi}
\makeatother
% Scale images if necessary, so that they will not overflow the page
% margins by default, and it is still possible to overwrite the defaults
% using explicit options in \includegraphics[width, height, ...]{}
\setkeys{Gin}{width=\maxwidth,height=\maxheight,keepaspectratio}
% Set default figure placement to htbp
\makeatletter
\def\fps@figure{htbp}
\makeatother
\setlength{\emergencystretch}{3em} % prevent overfull lines
\providecommand{\tightlist}{%
  \setlength{\itemsep}{0pt}\setlength{\parskip}{0pt}}
\setcounter{secnumdepth}{-\maxdimen} % remove section numbering
\ifLuaTeX
  \usepackage{selnolig}  % disable illegal ligatures
\fi

\title{Auto-Stats}
\author{Arne John}
\date{2022-06-18}

\begin{document}
\maketitle

\hypertarget{how-to-read-this-document}{%
\section{How to Read This Document}\label{how-to-read-this-document}}

Currently, this is a static html page for output. This document is a
report that becomes the output when a person is doing an analysis on
data.

Different types of text:

Regular text is the output found in the report, meant to be easily
readable/understandable for the user.

SUPERSCRIPT DENOTES REQUIRED INPUTS or extra output not created.
Addtionally, {[}Choices{]} are denoted in brackets for now.

{Red font is meant to show missing areas as reminders for me to improve
on.}

SELECT DATASET HERE. The independent Dataset is Kitchen Rolls; The
paired Dataset is Moon and Aggression.

\begin{Shaded}
\begin{Highlighting}[]
\CommentTok{\# This chunk is currently not being evaluated, as the data is being loaded in by the succeeding chunk. }
\CommentTok{\# Ultimately, the user should first upload a data file and then tell the program whether the data has paired samples or not. }
\CommentTok{\# The file is currently calling on two different datasets, depending on whether the user is analyzing paired or unpaired data.}
\NormalTok{data }\OtherTok{\textless{}{-}} \FunctionTok{read.csv}\NormalTok{(}\StringTok{"kitchenRollsTTest.csv"}\NormalTok{)}
\NormalTok{data\_paired }\OtherTok{\textless{}{-}} \FunctionTok{read.csv}\NormalTok{(}\StringTok{"Moon and Aggression.csv"}\NormalTok{)}
\end{Highlighting}
\end{Shaded}

Is dataset paired? Y/N {[}Y{]}

Does your alternative hypothesis assume that the mean is different,
greater, or lesser than the mean of the control group? {[}greater{]}

\begin{Shaded}
\begin{Highlighting}[]
\CommentTok{\# Since the moon dataset uses two columns and so combines observations, we need to tidy the data so each row is its own observation rather than participant. }
\NormalTok{data\_paired[}\StringTok{"Participants"}\NormalTok{] }\OtherTok{\textless{}{-}} \FunctionTok{rep}\NormalTok{(}\DecValTok{1}\SpecialCharTok{:}\DecValTok{15}\NormalTok{)}
\NormalTok{data\_paired }\OtherTok{\textless{}{-}} \FunctionTok{pivot\_longer}\NormalTok{(}\AttributeTok{data =}\NormalTok{ data\_paired,}
                            \AttributeTok{cols =}\NormalTok{ Moon}\SpecialCharTok{:}\NormalTok{Other,}
                            \AttributeTok{names\_to =} \StringTok{"Condition"}\NormalTok{,}
                            \AttributeTok{values\_to =} \StringTok{"Outbursts"}\NormalTok{)}
\end{Highlighting}
\end{Shaded}

SELECT DV AND IV. {[}In this case, IV need to be pairs: Moon \& Other{]}
{[}DV are all the cells responding to the columns, rows are
participants{]}

\begin{Shaded}
\begin{Highlighting}[]
\CommentTok{\# This code chunk represents the user\textquotesingle{}s choices, but since that will likely be done through JASP, it contains no actual code. }
\end{Highlighting}
\end{Shaded}

Maybe force the IV to be {[},1{]} and the DV to be {[},2{]} so the
following steps can just refer to those columns? Otherwise generalize it
so that following tests call upon the placeholder value.

Though this wouldn't work as for paired data, specific columns would be
selected to compare between.

\hypertarget{high-verbosity-hv}{%
\section{High Verbosity (HV)}\label{high-verbosity-hv}}

\hypertarget{introduction-what-are-you-achieving-with-this-test}{%
\subsubsection{Introduction: What are you achieving with this
test?}\label{introduction-what-are-you-achieving-with-this-test}}

Dear user, You selected to conduct a Student's t-test. This test checks
whether the data follows a Student's t-Distribution under the null
hypothesis. It is often used to see whether the means of two sets of
data are significantly different from one another.

\hypertarget{what-is-null-hypothesis-significance-testing-nhst}{%
\subsubsection{What is Null-Hypothesis Significance Testing
(NHST)?}\label{what-is-null-hypothesis-significance-testing-nhst}}

Why are we doing this? It's a comparison between means? But you could
also just estimate parameters

This procedure tests whether the group means are equal, but also the
size of the difference assumption it's not zero.

When researchers want to test whether some property or parameter has an
effect on the distribution of observed data, they conduct a null
hypothesis test. In this case, the null hypothesis assumes that the
parameter influencing the distribution has no effect, i.e.~is equal to
zero. In contrast, the alternative/statistical hypothesis calculates the
size of the effect on the distribution.

Statistical significance is asserted via the p-value. This value
calculated by the probability of obtaining a parameter that is at least
as extreme as the observed parameter, assuming the null hypothesis to be
correct/true. In other words, were one to repeat this experiment and
collect data each time, the chance of getting a test result that is as
high or higher as the observed result would only occur in p-value*100
percent of the time. Significance is assumed if the p-value falls below
some previously asserted threshold value, usually set to 0.05, also
known as the alpha level.

\hypertarget{descriptives}{%
\subsubsection{Descriptives}\label{descriptives}}

The data can be visualized with the help of a raincloud plot (Allen et
al., 2019). This allows an immediate intuitive assessment of whether the
groups differ, and whether the data in each group are approximately
normally distributed.

INSERT RAINCLOUD PLOT HERE

VISUALLY CONFIRMATION OF EQUAL VARIANCE BETWEEN DATA SETS \& OUTLIERS

Check for Model Misspecifications.

\hypertarget{assumption-checks}{%
\subsubsection{Assumption Checks}\label{assumption-checks}}

Before you can conduct a Paired t-test, we first need to check for
assumptions that need to be held in order for the test to be valid. The
assumptions for a Paired t-test are as follows:

\begin{itemize}
\tightlist
\item
  The two sets of data have similar variance.
\item
  There are no significant outliers in the data that could influence the
  results.
\item
  Neither data set contains skewness in the distribution of results.
\end{itemize}

Should independence be discussed? Can we test for independence in a
meaningful way, or is it just meant to be ``thought-provoking?''

\hypertarget{equality-of-variance-homoscedasity}{%
\paragraph{Equality of Variance /
Homoscedasity}\label{equality-of-variance-homoscedasity}}

{Paragraph explaining the necessity of equality of variance}

To test homogeneity of variance, Levene's test was performed. The test
did not show significance, meeting that the variances between the two
groups is not significantly different. This means that you can safely
conclude the results of the t-test.

\hypertarget{outliers}{%
\paragraph{Outliers}\label{outliers}}

{Do we want a paragraph on outliers, given that the raincloud plot is
above?}

It also assumes there to be no significant outliers that skew the
results. To test this, both a Grubb's test and a Dixon's Q test were
conducted. Both tests reported no significant effect of outliers, so you
can safely report the results of the test.

\hypertarget{skewness}{%
\paragraph{Skewness}\label{skewness}}

``Although paired t test are relatively robust to skewness, you could
either do a transformation (e.g.~log-transform), or consider to compare
the results to a rank-based test''

Last, a skewness test was conducted to see how non-symmetric your data
set is. The distribution of your data is moderately skewed, which means
you cannot assume a normality in your distribution. This is not
necessarily a problem, as long as your data set should be sufficiently
large (n per group \textless{} 20) to have a valid t-test result.
Alternatively, you run could a different test, such as INSERT TEST
ALTERNATIVE FOR SKEWED DATA

{Paragraph explaining the concept of skewness and its relevance for t
tests.}

\hypertarget{normality-test-shapiro-wilk}{%
\paragraph{Normality test
(Shapiro-Wilk)}\label{normality-test-shapiro-wilk}}

Paragraph explaining the concept of normality testing.

Needs updating for text.

\hypertarget{actual-test}{%
\subsubsection{Actual Test}\label{actual-test}}

Datasets: Is it possible to have a dataset on which one can test all
types of statistical tests? Can you reasonably run an independent and
paired t test on the same dataset?

Don't compare paired vs independent t tests.

{Have the user choose whether the alternative is two sided, less, or
greater, for purpose of demonstration it's less. TABLE DOESN'T WORK YET}

\begin{table}[tbp]

\begin{center}
\begin{threeparttable}

\caption{\label{tab:comparison_table}Paired Wilcoxon signed ranks test of the moon on outbursts}

\begin{tabular}{ll}
\toprule
$V$ & \multicolumn{1}{c}{$p$}\\
\midrule
119.00 & < .001\\
\bottomrule
\end{tabular}

\end{threeparttable}
\end{center}

\end{table}

\hypertarget{hypothesis-testing}{%
\subsubsection{Hypothesis Testing}\label{hypothesis-testing}}

The analysis shows that you have t(14) = 6.45 with p = 0. This means
that assuming the null hypothesis to be true, when running this test an
infinite amount of times the chances of getting a score of 6.45 or
higher would only occur in \ensuremath{8\times 10^{-4}}\% of cases.
Since 0 is

below the 0.05 threshold, the test is said to be significant and you can
reject the null hypothesis.

{Cannot just round, needs instead some evaluation like ``if p
\textless{} 0.001, print that''}

\hypertarget{misconceptions-on-hypothesis-testing-only-for-hv}{%
\subparagraph{Misconceptions on Hypothesis Testing; ONLY FOR
HV}\label{misconceptions-on-hypothesis-testing-only-for-hv}}

Unfortunately, it is easy to misconceptions about the p-value. In order
to avoid making incorrect assertions, take a look at the following
statements that are all incorrect:

\begin{itemize}
\tightlist
\item
  The p-value states the chance of the null hypothesis being true. So if
  a statistical analysis gives p = 0.034, then there is a 3.4\% chance
  of the null hypothesis being true.
\end{itemize}

Since the p-value is calculated under the assumption of the null
hypothesis being true, it logically cannot give any information on
whether the null hypothesis is true or not.

\begin{itemize}
\tightlist
\item
  The p-value denotes the probability of a type \(I\) error. So if a
  statistical analysis gives p = 0.034, then there is a 3.4\% chance of
  falsely rejecting the null-hypothesis.
\end{itemize}

While this statement on the surface looks compelling, it underlies the
same misconception of the previous statement. Since the p-value is
calculated under the assumption of the null hypothesis being true, it
cannot be linked to type \(I\) or \(II\) errors.

\begin{itemize}
\tightlist
\item
  A study with a lower p-value shows a more significant result than a
  study with a higher p-value. In other words, a study with p
  \textless{} 0.001 is more significant than a study with p \textless{}
  0.05.
\end{itemize}

The p-value does not give any information on the significance of
whatever was being tested. It only conveys statistical significance in
the sense that the value is below a threshold (often set to 0.05), but
it has no connection to effect size.

\begin{itemize}
\tightlist
\item
  A p-value of 0.05 means that we have observed data that would occur
  only 5\% of the time under the null hypothesis.
\end{itemize}

This cannot be the case if you remember that the definition of the
p-value is the probability of the observed data \textless span
font-style: italic;\textgreater plus more extreme data assuming the null
hypothesis to be true.

\hypertarget{parameter-estimation}{%
\paragraph{Parameter Estimation}\label{parameter-estimation}}

Now that you confirmed the assumptions to hold, you conducted a Paired
t-test to test whether the mean of the differences is greater than the
NA. To conclude this, you want to reject the null-hypothesis that
assumes the true difference in means to be 0.

Specifically, you test for a difference in Outbursts by Condition with a
95\% confidence interval.

\hypertarget{misconceptiions-about-parameter-estimation-only-for-hv}{%
\subparagraph{Misconceptiions about Parameter Estimation; ONLY FOR
HV}\label{misconceptiions-about-parameter-estimation-only-for-hv}}

In a similar vein as the misconceptions about the t-test for hypothesis
testing, here are some statements of seemingly persuasive arguments
which nonetheless are all false.

\begin{itemize}
\tightlist
\item
  You should use a one-sided p-value when you don't care about a result
  in one direction, or that direction makes no conceptual sense.
\end{itemize}

This might come as a surprise, as the idea of a one-sided test seems
appropriate to avoid ``false'' probabilities by including
non-nonsensical possibilities. However, deciding on a one-sided test
mathematically strengthens the evidence found in the tested direction,
effectively allowing the beliefs and attitudes of the researcher to
influence the statistical result.

If you want to learn more about this issue, check out

\hypertarget{limitations-of-nhst}{%
\subsubsection{Limitations of NHST}\label{limitations-of-nhst}}

There exist several criticisms to this method to determine the
significance of an effect.

\begin{itemize}
\tightlist
\item
  it never actually tests the statistical hypothesis.
\item
  The threshold of 0.05 is arbitrary
\item
  due to its method, interpretations are often incorrect as they
  forget/falsely assert the meaning of the probability being dependent
  on the null hypothesis being true
\item
  requires blindness of the researchers on the data to avoid bias.
\item
\end{itemize}

\hypertarget{sourcesreferences}{%
\section{Sources/References}\label{sourcesreferences}}

Fisher, R (1955). ``Statistical Methods and Scientific Induction''.
Journal of the Royal Statistical Society, Series B. 17 (1): 69--78.

Neyman, J; Pearson, E. S. (January 1, 1933). ``On the Problem of the
most Efficient Tests of Statistical Hypotheses''. Philosophical
Transactions of the Royal Society A. 231 (694--706): 289--337.

\hypertarget{data}{%
\subsection{Data}\label{data}}

Wagenmakers, E.-J., Beek, T. F., Rotteveel, M., Gierholz, A., Matzke,
D., Steingroever, H., \ldots{} Pinto, Y. (2015). Turning the hands of
time again: A purely confirmatory replication study and a Bayesian
analysis. Frontiers in Psychology, 6.

Moore, D. S., McCabe, G. P., and Craig. B. A. (2012) Introduction to the
Practice of Statistics (7th ed). New York: Freeman.

\hypertarget{tests}{%
\subsection{Tests}\label{tests}}

Mann-Whitney Non-Parametric Test / Wilcoxon Rank Sum

David F. Bauer (1972). Constructing confidence sets using rank
statistics. Journal of the American Statistical Association 67,
687--690. doi: 10.1080/01621459.1972.10481279.

Myles Hollander and Douglas A. Wolfe (1973). Nonparametric Statistical
Methods. New York: John Wiley \& Sons. Pages 27--33 (one-sample), 68--75
(two-sample). Or second edition (1999).

Shapiro-Wilk Normality Test

Patrick Royston (1982). An extension of Shapiro and Wilk's WW test for
normality to large samples. Applied Statistics, 31, 115--124.
\url{doi:10.2307/2347973}.

Levene's test

Fox, J. and Weisberg, S. (2019) An R Companion to Applied Regression,
Third Edition, Sage.

Raincloud Plot

Allen, M., Poggiali, D., Whitaker, K., Marshall, T. R., \& Kievit, R. A.
(2019). Raincloud plots: a multi-platform tool for robust data
visualization. Wellcome open research, 4, 63.
\url{https://doi.org/10.12688/wellcomeopenres.15191.1}

Student's t-test

``Student'' William Sealy Gosset (1908). ``The probable error of a
mean'' (PDF). Biometrika. 6 (1): 1--25. \url{doi:10.1093/biomet/6.1.1}.
hdl:10338.dmlcz/143545.

Welch's t-test

Welch, B. L. (1947). ``The generalization of''Student's'' problem when
several different population variances are involved''. Biometrika. 34
(1--2): 28--35. \url{doi:10.1093/biomet/34.1-2.28}

\end{document}
